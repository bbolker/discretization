% Options for packages loaded elsewhere
\PassOptionsToPackage{unicode}{hyperref}
\PassOptionsToPackage{hyphens}{url}
%
\documentclass[12pt]{article}
\usepackage{times}
\usepackage{amsmath,amssymb}
\usepackage{url}
\usepackage{iftex}
\ifPDFTeX
  \usepackage[T1]{fontenc}
  \usepackage[utf8]{inputenc}
  \usepackage{textcomp} % provide euro and other symbols
\else % if luatex or xetex
  \usepackage{unicode-math} % this also loads fontspec
  \defaultfontfeatures{Scale=MatchLowercase}
  \defaultfontfeatures[\rmfamily]{Ligatures=TeX,Scale=1}
\fi
\usepackage{lmodern}
\ifPDFTeX\else
  % xetex/luatex font selection
\fi
% Use upquote if available, for straight quotes in verbatim environments
\IfFileExists{upquote.sty}{\usepackage{upquote}}{}
\IfFileExists{microtype.sty}{% use microtype if available
  \usepackage[]{microtype}
  \UseMicrotypeSet[protrusion]{basicmath} % disable protrusion for tt fonts
}{}
\makeatletter
\@ifundefined{KOMAClassName}{% if non-KOMA class
  \IfFileExists{parskip.sty}{%
    \usepackage{parskip}
  }{% else
    \setlength{\parindent}{0pt}
    \setlength{\parskip}{6pt plus 2pt minus 1pt}}
}{% if KOMA class
  \KOMAoptions{parskip=half}}
\makeatother
\usepackage{xcolor}
\setlength{\emergencystretch}{3em} % prevent overfull lines
\providecommand{\tightlist}{%
  \setlength{\itemsep}{0pt}\setlength{\parskip}{0pt}}
\setcounter{secnumdepth}{-\maxdimen} % remove section numbering
\ifLuaTeX
  \usepackage{selnolig}  % disable illegal ligatures
\fi
\usepackage{bookmark}
\IfFileExists{xurl.sty}{\usepackage{xurl}}{} % add URL line breaks if available
\urlstyle{same}
\hypersetup{
  pdftitle={response to reviewer comments},
  hidelinks,
  pdfcreator={LaTeX via pandoc}}

\title{response to reviewer comments}
\author{}
\date{}

\begin{document}
\maketitle

\subsection{reviewer 1}\label{reviewer-1}

There's a lot to like about this paper. Who doesn't enjoy a polemic? Who
doesn't want a simple answer to a complicated problem. Ultimately,
however, I don't think this paper works as a standalone paper. Either,
it could serve as the basis of a read paper with rejoinders and further
discussion. Or it could be expanded under the more honest title: ``There
is no best way to understand multifactorial systems''.

Doing statistics is hard. Learning from data is hard. To deny this is to
show that one has never had to analyse real observational data and that
one thinks generations of statisticians have been severely misguided as
to what they should be doing.

\emph{I don't think I ever denied this, explicitly or implicitly
(although I admit I may have been overstating things a bit for
rhetorical effect)}

The view that the best approach is ``to use full (maximal) statistical
models'' is, as the author is no doubt fully aware, completely
misleading. It fails spectacularly in two very common scenarios: (1)
when there are correlated explanatory variables, and (2) when the effect
on the outcome of interest is nonlinear. To take the author's own
example: precipitation and temperature are correlated variables whose
effect is often nonlinear. When they are correlated, the full model can
easily be too complicated and inferred effects can have the wrong sign
and/or inflated CIs; when their effect is nonlinear, the full model is
too simple.

\emph{I either don't understand the reviewer's point completely, or I
disagree. Since an additional reviewer also raised the point of
collinearity, I added a new paragraph that goes into more detail (and
provides multiple citations) and argues that using the full model with
collinear variables does not ``fail spectacularly''; rather, it gives an
accurate assessment of the degree of uncertainty about the strengths of
the correlated effects. I don't understand what issue with nonlinearity
is. My suggestion to use the ``full model'' does not mean we are
restricted to linear terms.}

And this raises another issue with the paper: the class of models the
author seems to be contemplating is rather limited. Why not use
generalised additive models? Random forests? There is a whole raft of
approaches to prevent overfitting, to determine the relative importance
of explanatory variables, etc that has real relevance to the question of
how to understand multifactorial systems.

\emph{I never intended the class of models to be restricted. I added two
points to address this: (1) state in the first paragraph that the
emphasis is on inference and quantification rather than prediction; (2)
add references to additive models and Gaussian processes under the
section on penalization.}

Finally, there is no discussion (or criticism) in the paper of the
Gelman \& Hill approach to model building. I think this has particular
relevance to the ecological applications that the author has in mind.

\emph{While I am familiar with Gelman and Hill's book in general, I
don't know exactly what the ``Gelman \& Hill'' approach is. I've added a
reference to Gelman and Shalizi discussing the idea of continuous model
expansion.}

Minor comments:

\begin{itemize}
\tightlist
\item
  An actual working example of the author's claims would have aided this
  paper enormously.
\end{itemize}

\emph{Didn't have time to address this in the amount of time allowed for
the revision (10 days).}

\begin{itemize}
\tightlist
\item
  Line 34: Please can we stop advocating that computing R\^{}2 is a
  measure of whether our model ``describes the data reasonably well''.
  Perhaps more important here is to stress the importance of the
  checking of \emph{model assumptions}.
\end{itemize}

\emph{this line changed to ``e.g.~by examining model diagnostics and by
making sure that the level of unexplained variation is not unacceptably
large''}

\begin{itemize}
\tightlist
\item
  Line 103: I am not sure what is meant by ``unnecessarily discretizing
  a continuous world'' in this context. Is the author suggesting model
  space is continuous? But isn't that what MMA is trying to recover? Or
  is the author saying that it is a category error to imagine the
  components of a multifactorial system have independent effects? But in
  that case a multifactorial model can only ever be a reflection of the
  imaginative power of the researcher. Who is to say what the ``full
  statistical model'' is?
\end{itemize}

\emph{I have changed this to ``a continuous model space''. MMA is indeed
trying to recover a continuous model space, but my whole point is that
doing it by averaging discrete models is a detour.}

Miscellaneous:

\begin{itemize}
\tightlist
\item
  Line 193: The author needs to state when this Google Scholar search
  was performed.
\end{itemize}

\emph{Results updated and dated}

\pagebreak

\subsection{reviewer 2}\label{reviewer-2}

An interesting paper with some useful insights. I share a few thoughts
and critiques below. Lines 11-12: Perhaps some rewording required here
-- (1) aren't all models artificial? Are you trying to say models of
little interest?

\emph{Replaced with ``artificially simplified''}

Lines 13-16: Isn't all shrinkage going to be dependent on sample size
relative to number of parameters in the model? Larger sample sizes
relative to number of parameters resulting in less shrinkage.

\emph{Yes, but I don't see why this is particularly relevant to the
abstract? added ``relative to the amount of data''}

Lines 31-44: It seems as if you are equating multiple processes with
multiple variables in a model. Do your arguments still apply in a
situation where multiple variables are used to characterize one
encompassing process? Or at least where there is not a one-one mapping
of processes and variables?

\emph{Yes, they do. I'm having trouble finding/understanding which
passage this refers to, or why the reviewer would draw this conclusion
\ldots{}}

Line 66: Insert word ``it'', ``unequivocally that it has''.

\emph{Done.}

Lines 72-77: One issue with only fitting the full model with all factors
simultaneously even with very large sample sizes, is that the magnitude
of the parameter estimates for each factor (variable) will reflect the
multicollinearity among all the factors (variables) which alters their
interpretation compared to a model with no collinearity among the
factors (variables). See Cade (2015). There almost always is substantial
multicollinearity among factors (predictor variables) in ecological
models. One approach to help understand this better might be to not only
consider the full model with all factors that has parameter estimates
that reflect multicollinearity among factors (predictor variables), but
to consider single factor (predictor) models where there is no
collinearity possible. This could be used as a device to help understand
how much rates of change associated with factor parameters have been
altered by multicollinearity with other factors in the full model. It is
easy to forget that regression coefficients estimated in a multiple
regression model only provide an interpretation for the part of a factor
(predictor variable) that is not linearly related to the other factors
(predictor variables) as Cade (2015) points out. If we really want to
know about changes in outcomes associated with the linearly correlated
parts of multiple factors, then it seems like something else must be
done. It seems to me that the crucial exploration of models with
multiple factors that have some degree of collinearity (like
precipitation and temperature in the example on lines 104-105), is to
understand changes in outcomes with changes in factors that occur
simultaneously. This will usually require more than just simple
interpretation of individual parameter estimates in a multifactorial
model when there is some degree of multicollinearity among the factors.

\emph{I've added a whole paragraph about collinearity to address this
issue.}

Line 88: Seems like Cade (2015) critique about MMA of coefficients is
principally about issues associated with collinear predictor variables.
The critique of Walker (2017) relies on fraught causal interpretations
of regression models that is inconsistent with conventional statistical
theory associated with regression models. I would eliminate this
citation.

\emph{Done.}

Lines 89-90: Seems like this needs to be reorganized a bit. The issue
with variable collinearity (multicollinearity) is directly related to
issues with model averaging estimates of predictor parameters
(regression coefficients).

\emph{See reorganization/extended discussion of collinearity}

Lines 90-91: Again, Cade (2015) critique also applies to problems with
summing model weights to assess relative importance of predictors.

\emph{Added.}

Line 109: We could fit all five of these models and NOT average their
parameters.

\emph{I'm not sure what the proposal is here. How exactly would we
compare?}

Line 149-150: There are several built in approaches to obtaining
shrinkage estimates for quantile regression. The lasso is an option in
Roger Koenker's quantreg package and other shrinkage type estimators are
available in Fasiolo's qgam package.

\emph{Added a reference to Koenker's review paper which mentions
penalization}

Lines 155-161: I think I recall reading somewhere that with lasso
shrinkage estimates that it may be appropriate to reestimate a model
that excludes all the parameters that were shrunk to zero to obtain
better standard errors. I would guess this might have something to do
with correctly reflecting the multicollinearity among predictors. Can you
offer any comments or insights on this model reestimation approach after
shrinkage?

{\em This approach was originally proposed by Efron et al.~(2004) as a
``LARS-OLS hybrid'' and generalized by Meinshausen 2007 as the ``relaxed
lasso''. I don't think this helps with collinearity issues (the elastic
net does, though). While interesting, I think this is out of scope for
this paper.

Efron, Bradley, Trevor Hastie, Iain Johnstone, and Robert Tibshirani.
2004. ``Least Angle Regression.''
\url{https://projecteuclid.org/journals/annals-of-statistics/volume-32/issue-2/Least-angle-regression/10.1214/009053604000000067.short}.

Meinshausen, Nicolai. 2007. ``Relaxed Lasso.'' Computational Statistics
\& Data Analysis 52 (1): 374--93.
\url{https://doi.org/10.1016/j.csda.2006.12.019}.
}

Line 199: Disagree. Some things we can imagine have no influence on the
outcomes we observe. They may sometimes be weakly correlated with other
relevant processes.

\emph{I have softened this statement: ``In psychology, economics,
sociology, epidemiology, ecology, and evolution, most processes that we
would think of including in a model have \emph{some} influence on the
outcomes that we observe. Pretending that some of these processes are
completely absent can be a useful means to an inferential or
computational end, but it is rarely what we actually believe about the
system (although see @mundryIssues2011 for a counterargument)''}

\pagebreak

\subsection{reviewer 3}\label{reviewer-3}

The manuscript discusses the limitations of multi-model approaches in
dealing with multifactorial systems. The author presents several
interesting observations that merit publication. Particularly noteworthy
is the discussion regarding the potential pitfalls of model averaging.
However, I believe the paper could benefit from improved precision in
certain statements and increased accessibility for non-ecologists.

\begin{itemize}
\tightlist
\item
  The paragraph starting at line 43: Some of the philosophical
  statements in this paragraph may be subject to scrutiny. Specifically,
  it appears as though the author is asserting that only the ``full
  model'' is reasonable to consider. The notion of a full model is also
  referenced later in line 135, where the author seems to assume that
  analysts typically have a full model that closely approximates truth.
  I am uncertain of the author's intended meaning, but it seems these
  statements require clarification. It may be beneficial to express these
  ideas with greater precision, or if not possible, to soften them. It
  is my understanding that constructing a complete model for complex
  multifactorial phenomena is often unattainable. We operate under the
  assumption that some factors may explain observations, yet inevitably
  overlook others. Thus, all models can be regarded as subsets of others
  that may better explain the data, unrecognized to the analyst.
  Consequently, it may be prudent to explore models that encompass only
  a subset of factors present in the largest model conceivable by the
  analyst.
\end{itemize}

\emph{I've expanded this paragraph to try to clarify that while the
AIC-optimal model is the closest predictive model to the truth, we don't
usually believe that processes are dropped from the model because they
don't exist, but rather because they can't be estimated reliably (so
don't add to predictive accuracy).}

Firstly, parsimony is crucial to mitigate overfitting -- this is somehow
mentioned by the author. Secondly, selecting appropriate models is
largely empirical, and I am reluctant to dissuade scientists from
considering simpler models if they adequately explain observations
compared to more complex alternatives.

\emph{Points taken, but this should generally be done a priori - if it's
done in a data-driven way it makes it hard to construct reliable
confidence intervals}

\begin{itemize}
\tightlist
\item
  Does the paper address the possibility of correlations between
  different factors in certain applications? This consideration seems
  significant and warrants discussion.
\end{itemize}

\emph{I've added a paragraph about collinearity}

\begin{itemize}
\tightlist
\item
  Line 78: Stepwise regression is described as deprecated, although I
  guess it may still be widely utilized in certain contexts---perhaps
  not so much within ecology?
\end{itemize}

\emph{It is certainly still used in some fields of \textbf{predictive}
modeling, although I would personally argue that it is dominated (where
computational constraints allow) by forms of penalized regression that
allow simultaneous modeling of the whole data set. I added references to
more critiques.}

While I acknowledge that data constraints can severely impact the efficacy
of stepwise regression, many criticisms of this method presuppose
knowledge of the correct model.

\emph{Yes, because without an analysis based on a knowledge of a
generating model, it is hard to evaluate whether one is getting
inferences correct}

However, this is rarely the case when proposing models to explain a
dataset. Consequently, what purpose is served by employing a more
complex model if the available data can be explained by a simpler one?
The simpler model may exhibit bias compared to a superior model fitted to
a larger dataset. Nonetheless, such a comparison is only feasible when a
larger dataset and a more appropriate model are available.

\emph{The argument is not that we know the true model, but that the
simplfied models are pretending that some processes are exactly zero.
This is a conceptual issue we should be aware of, even if a reduced
model is ``good enough'' (and likely provides better predictive
accuracy).}

\begin{itemize}
\tightlist
\item
  In relation to the previous point, does the author consider stepwise
  regression based on combinations of multiple factor subsets or only
  one factor at a time? For instance, Hastie et al.'s ``The Elements of
  Statistical Learning'' advocate for exploring multiple subsets.
\end{itemize}

\emph{This may be a useful approach to predictive modeling, but I don't
see it as being useful for constructing an inferential model. In the
paragraph on correlation I added a description of simultaneously
dropping several factors to evaluate their joint importance.}

\begin{itemize}
\tightlist
\item
  Incidentally, adding a citation to Hastie et al.'s book may be
  appropriate regarding penalized approaches.
\end{itemize}

\emph{Done}

\begin{itemize}
\tightlist
\item
  I am concerned that the paper's key messages may not be fully
  appreciated by many readers. While there is a discernible emphasis on
  ecology, readers outside this domain may struggle to extract relevant
  information. Accordingly, I propose several modifications and
  additions: o Introducing the example from the paragraph starting at
  line 103 earlier in the paper and utilizing it throughout to elucidate
  various concepts.
\end{itemize}

\emph{Good idea, done}

\begin{itemize}
\item Providing a straightforward illustration of multi-model averaging
would help.
\item A diagram or table summarizing the discussed methods and
their respective strengths and weaknesses could prove beneficial.
\end{itemize}

\emph{These are good ideas, but I haven't done them; there are many examples of multi-model averages in the references (e.g. in Burnham and Anderson's books)}

\end{document}
